\documentclass[]{article}
\usepackage{lmodern}
\usepackage{amssymb,amsmath}
\usepackage{ifxetex,ifluatex}
\usepackage{fixltx2e} % provides \textsubscript
\ifnum 0\ifxetex 1\fi\ifluatex 1\fi=0 % if pdftex
  \usepackage[T1]{fontenc}
  \usepackage[utf8]{inputenc}
\else % if luatex or xelatex
  \ifxetex
    \usepackage{mathspec}
  \else
    \usepackage{fontspec}
  \fi
  \defaultfontfeatures{Ligatures=TeX,Scale=MatchLowercase}
\fi
% use upquote if available, for straight quotes in verbatim environments
\IfFileExists{upquote.sty}{\usepackage{upquote}}{}
% use microtype if available
\IfFileExists{microtype.sty}{%
\usepackage{microtype}
\UseMicrotypeSet[protrusion]{basicmath} % disable protrusion for tt fonts
}{}
\usepackage[margin=1in]{geometry}
\usepackage{hyperref}
\hypersetup{unicode=true,
            pdftitle={R for the Student},
            pdfauthor={JD Long \& Dusty Turner},
            pdfborder={0 0 0},
            breaklinks=true}
\urlstyle{same}  % don't use monospace font for urls
\usepackage{biblatex}

\addbibresource{bibliography.bib}
\usepackage{graphicx,grffile}
\makeatletter
\def\maxwidth{\ifdim\Gin@nat@width>\linewidth\linewidth\else\Gin@nat@width\fi}
\def\maxheight{\ifdim\Gin@nat@height>\textheight\textheight\else\Gin@nat@height\fi}
\makeatother
% Scale images if necessary, so that they will not overflow the page
% margins by default, and it is still possible to overwrite the defaults
% using explicit options in \includegraphics[width, height, ...]{}
\setkeys{Gin}{width=\maxwidth,height=\maxheight,keepaspectratio}
\IfFileExists{parskip.sty}{%
\usepackage{parskip}
}{% else
\setlength{\parindent}{0pt}
\setlength{\parskip}{6pt plus 2pt minus 1pt}
}
\setlength{\emergencystretch}{3em}  % prevent overfull lines
\providecommand{\tightlist}{%
  \setlength{\itemsep}{0pt}\setlength{\parskip}{0pt}}
\setcounter{secnumdepth}{0}
% Redefines (sub)paragraphs to behave more like sections
\ifx\paragraph\undefined\else
\let\oldparagraph\paragraph
\renewcommand{\paragraph}[1]{\oldparagraph{#1}\mbox{}}
\fi
\ifx\subparagraph\undefined\else
\let\oldsubparagraph\subparagraph
\renewcommand{\subparagraph}[1]{\oldsubparagraph{#1}\mbox{}}
\fi

%%% Use protect on footnotes to avoid problems with footnotes in titles
\let\rmarkdownfootnote\footnote%
\def\footnote{\protect\rmarkdownfootnote}

%%% Change title format to be more compact
\usepackage{titling}

% Create subtitle command for use in maketitle
\providecommand{\subtitle}[1]{
  \posttitle{
    \begin{center}\large#1\end{center}
    }
}

\setlength{\droptitle}{-2em}

  \title{R for the Student}
    \pretitle{\vspace{\droptitle}\centering\huge}
  \posttitle{\par}
    \author{JD Long \& Dusty Turner}
    \preauthor{\centering\large\emph}
  \postauthor{\par}
    \date{}
    \predate{}\postdate{}
  

\begin{document}
\maketitle

output: html\_document: citation\_package: biblatex keep\_tex: TRUE
bibliography: bibliography.bib biblio-style: authoryear csl:
chicago-note-bibliography.csl

\subsection{Dusty's Thoughts to flesh out
somewhere}\label{dustys-thoughts-to-flesh-out-somewhere}

testing citation \autocite*{R-knitr} more typing

testing citation \autocite*{R-rmarkdown} more typing

testing citation \textcite{R-rmarkdown} cited inline

When teaching, is the intent to teach coding in R, or is it to use R as
an aide to teach statistics? This will drive what you do and why you do
it.

\subsubsection{What is the intent of teaching with
R?}\label{what-is-the-intent-of-teaching-with-r}

I have taught advanced introduction to probability and statistics and
used R as a tool to teach statistics. We introduced it by showing how it
can be a good calculator -- and along the way taught some of the R
Studio functionality (projects/setting working directory, how to execute
lines of code, save variables, etc).

I have also taught sabermetrics where we taught coding in R. So we
needed to do data analysis (and we worked under the assumptions that our
cadets knew basic statistics) so I taught the coding as a means of data
analysis.

\subsubsection{\texorpdfstring{Do you ``give stutents
code''?}{Do you give stutents code?}}\label{do-you-give-stutents-code}

You have two options:

\begin{enumerate}
\def\labelenumi{\arabic{enumi})}
\tightlist
\item
  Give students code before class.
\end{enumerate}

The advantage here is that you can talk concepts and the syntax and
inevitable mistyping, etc does not inhibit the larger purpose of the
instruction.

The disadvantage is that students don't get intimate with the coding
process and learn the necessary skill of trial and error.

\begin{enumerate}
\def\labelenumi{\arabic{enumi})}
\setcounter{enumi}{1}
\tightlist
\item
  Don't give them code and expect them to keep up.
\end{enumerate}

The advantage is that cadets are learning the process of coding and
``own'' the code while they write it.

The disadvantage is instead of teaching concepts, students are
distracted by keeping up with the coding and missing out on the
statistics lesson

My recommendation -- you give code if your intent is teaching
statistics. You don't give code if the intent is that cadets learn to
code.

\subsection{Abstract}\label{abstract}

\subsection{Introduction}\label{introduction}

\subsection{The R Ecosystem}\label{the-r-ecosystem}

\subsubsection{CRAN}\label{cran}

\subsubsection{R Studio}\label{r-studio}

\subsubsection{Tidyverse}\label{tidyverse}

\subsection{Installation}\label{installation}

paraphrase what's in R Cookbook?

\subsubsection{Installing R}\label{installing-r}

\subsubsection{Installing R Studio}\label{installing-r-studio}

\subsubsection{Installing the Tidyverse}\label{installing-the-tidyverse}

\subsection{Using R}\label{using-r}

\subsubsection{Loading Data}\label{loading-data}

\subsubsection{Plotting}\label{plotting}

quick intro to ggplot

That graphical tool for ggplot?

\subsubsection{Plotting Multivariate
Data}\label{plotting-multivariate-data}

ggpairs?

\section{Bibliography}\label{bibliography}

\textcite{R-knitr} \textcite{R-rmarkdown}

\printbibliography


\end{document}
